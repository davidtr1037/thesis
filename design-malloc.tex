%!TEX root=./paper.tex

\subsection{Memory allocations}
\label{Se:Malloc}

Let us consider the example from
Figure~\ref{fig:slices-with-allocations}, where the skipped function
\code{f} allocates memory with \code{malloc}. After skipping the
function call at line~\ref{line:alloc_skip}, the chopped symbolic
execution encounters two dependent loads at
lines~\ref{line:alloc_first} and \ref{line:alloc_second}. Chopped
symbolic execution thus spawns two consecutive recovery states: one in
which executes only line~\ref{line:slice_1}, and one in which executes lines~\ref{line:slice_1} and
lines~\ref{line:slice_2}. If we allowed \code{malloc} to return two different addresses while
executing the recovery states, this may lead to undefined behavior
since the second recovery would write to a different memory
address. To prevent this, and maintain consistency across recovery
states originating from the same function call, we maintain a list of
returned addresses for each allocation site in \code{f} which are
identified by their call stack. This way, subsequent recovery states
will use this information while re-executing allocating instructions.

\begin{figure}[tbp]
  \lstinputlisting[linewidth=.95\columnwidth]{code/allocations.c}\vspace{-2mm}
  \caption{Example of skipped function with allocation.}\vspace{-4mm}
\label{fig:slices-with-allocations}
\end{figure}

%%% Local Variables:
%%% mode: latex
%%% TeX-master: "paper"
%%% End:
