%% $Id: thesis.tex,v 1.16 2004/09/01 14:37:00 yahave Exp $

%%%%%%%%%%%%%%%%%%%%%%%%%%%%%%%%%%%%%%%%%%%%%%%%%%%%%%%%%%%%%%%%%%%%%%
%%
%%  UT-THESIS.TEX
%%  Copyright (c) 1999 by Francois Pitt
%%  Last Update: 1999 May 13
%%
%%%%%%%%%%%%%%%%%%%%%%%%%%%%%%%%%%%%%%%%%%%%%%%%%%%%%%%%%%%%%%%%%%%%%%
%%
%%  This file is distributed in the hope that it will be useful but
%%  without any warranty (without even the implied warranty of
%%  fitness for a particular purpose).  For a description of this
%%  file's purpose, and instructions on its use, see below.
%%
%%  Feel free to copy and redistribute this file, as long as this
%%  copyright notice remains intact and this file is distributed
%%  along with the companion file `ut-thesis.cls'.
%%
%%  (Thanks to Robert Bernecky for his suggestions on improving the
%%  usefulness and readability of this file.)
%%
%%  Send all bugs, questions, comments, suggestions, etc. to the
%%  author, at <fpitt@cs.utoronto.ca>.
%%
%%%%%%%%%%%%%%%%%%%%%%%%%%%%%%%%%%%%%%%%%%%%%%%%%%%%%%%%%%%%%%%%%%%%%%
%%
%%  Skeleton LaTeX2e file for the preparation of theses at UofT;
%%  conforms to the School of Graduate Studies' guidelines of 07/97.
%%  To be used in conjunction with class file `ut-thesis.cls', whose
%%  features it illustrates.
%%
%%  To comment out parts of a file, use the macro \ignore{...}
%%  around the entire block of text you want to ignore.
%%
%%  To explicitly set the pagestyle of any inserted blank page when
%%  \cleardoublepage occurs, use one of \clearemptydoublepage or
%%  \clearplaindoublepage instead.
%%
%%  For single-spaced quotes or quotations, use the `longquote' and
%%  `longquotation' environments.  For single-spaced, 1 1/2-spaced,
%%  or double-spaced paragraphs, use one of the environments
%%  `singlespaced', `oneandahalfspaced', or `doublespaced'.  More
%%  generally, for paragraphs with a line spacing of `n', use
%%  `\begin{newspacing}{n}...\end{newspacing}'.
%%
%%  All other environments, commands, and options provided by the
%%  `ut-thesis' class will be described below, at the point where
%%  they should appear in the document.
%%
%%  See the companion file `ut-thesis.cls' for more details.
%%
%%%%%%%%%%%%%%%%%%%%%%%%%%%%%%%%%%%%%%%%%%%%%%%%%%%%%%%%%%%%%%%%%%%%%%


%%%%%%%%%%%%         PREAMBLE         %%%%%%%%%%%%

%% Default settings format a final copy (12pt font, single-sided,
%% double-spaced, normal margins, single-spaced notes).  For a rough
%% copy (10pt font, double-sided, double-spaced, normal margins, with
%% the word "DRAFT" printed at each corner of every page), use the
%% `draft' option.  The default line spacing can be changed with one
%% of the following options: `singlespaced', `oneandahalfspaced', or
%% `doublespaced'.  The notes are always single-spaced by default, but
%% can be made to have the same spacing as the rest of the document by
%% using the option `spacednotes'.  The size of the margins can be
%% changed with one of the following options: `narrowmargins' (1 1/4"
%% left, 3/4" others), `normalmargins' (1 1/4" left, 1" others),
%% `widemargins' (1 1/4" all), `extrawidemargins' (1 1/2" all).  Any
%% other standard option for the `report' document class can be used
%% to override the default or draft settings.

%% ***   Add any desired options.   ***
%%\documentclass{ut-thesis}

%% production
\newcommand{\myDocOptions}{11pt,twoside,openright,oneandahalfspaced,normalmargins}
\documentclass[pdftex,\myDocOptions]{ut-thesis}


\newcommand{\myTitle}{Chopped Symbolic Execution}
\newcommand{\myAuthor}{David Trabish}


%% ***   Add \usepackage declarations here.   ***

\pdfoutput=1
\usepackage[pdftex,bookmarks=true,colorlinks=true, pdfstartview=FitV, linkcolor=blue,
         citecolor=blue, urlcolor=blue,plainpages=false]{hyperref}
\usepackage[pdftex]{graphicx}
\pdfinfo{
    /Title    (\myTitle)
    /Author   (\myAuthor)
    /Keywords (Symbolic Exwcution, Static Analysis, Compilers, Program Analysis)
    /Keywords ()
}


\usepackage{amsthm}
\usepackage{amsfonts}
\usepackage{amssymb}
\usepackage{alltt}
%\usepackage{graph}
\usepackage{times}
\usepackage{fancyvrb}
\usepackage{import}
\usepackage{epsfig}
\usepackage{stmaryrd}
\usepackage{graphicx}
\usepackage{enumerate}
\usepackage[usenames]{color}
\usepackage{makeidx}
\usepackage[caption=false]{subfig}
\usepackage{longtable}
\usepackage{floatrow}

% customized packages

\usepackage{booktabs} % For formal tables

\usepackage{amsmath}
\usepackage{MnSymbol}
%\usepackage{amssymb}
%\usepackage{stmaryrd}
\usepackage{array}
\usepackage{listings}
\usepackage[shadow]{todonotes}
\usepackage{xspace}
\usepackage{paralist}
\let\labelindent\relax
\usepackage{enumitem}
\usepackage{multirow}
\usepackage{xcolor,colortbl}
\usepackage{flushend}
\usepackage{makecell}
\usepackage{arydshln}
\usepackage{mdframed}
\usepackage{flushend}
%\usepackage[utf8]{inputenc}
\usepackage[T1]{fontenc}
\usepackage{microtype}
\usepackage{algpseudocode}
\usepackage{algorithm}
\usepackage{alltt}
%\usepackage{amsmath,MnSymbol}% ,stmaryrd,amssymb,amsfonts,MnSymbol}
\usepackage{extarrows}
\usepackage{stackrel}
\usepackage{stfloats} % Fixes problem with latex2e figures
\usepackage{makecell}
\usepackage{graphicx}
\usepackage{adjustbox}
\usepackage{verbatim}

\usepackage{datetime}
\usepackage{todonotes}

\usepackage{cleveref} % Must be included last


\lstset{
language=C,
mathescape=true,
numbers=left,
stepnumber=1,
numberblanklines=false,
basicstyle=\footnotesize,
numberstyle=\footnotesize,
numbersep=3pt,
escapeinside={/*}{*/},
}

\makeatletter
\@addtoreset{example}{chapter}
\@addtoreset{lemma}{chapter}
\@addtoreset{theorem}{chapter}
\makeatother

\theoremstyle{definition}
\newtheorem{example}{Example}[section]
\newtheorem{lemma}{Lemma}[section]
\newtheorem{theorem}{Theorem}[section]
%% The line spacing of the document should be specified using one of
%% the document options given above, but if you need a line spacing
%% that is not provided by the options, you can override the default
%% line spacing for the entire document with the command
%%   `\linespacing{...}'.
%% Note that in order to get the correct appearance, the argument to
%% `\linespacing' must be equal to 1/3 + 2/3 times the desired line
%% spacing (for example, single-spaced = \linespacing{1},
%%                        1 1/2-spaced = \linespacing{1.33}, and
%%                       double-spaced = \linespacing{1.66}).

%% ***   Uncomment and fill in a value, if needed.    ***
%% ***   REMEMBER: You should NOT need to use this.  Use one of   ***
%% ***   the document class options mentionned above instead.     ***
%\linespacing{}

%%%%%%%%%%%%%%%%%%%%%%%%%%%%%%%%%%%%%%%%%%%%%%%%%%%%%%%%%%%%%%%%%%%%%%
%%                                                                  %%
%%                  ***   I M P O R T A N T   ***                   %%
%%                                                                  %%
%%  Fill in the following fields with the required information:     %%
%%   - \degree{...}       name of the degree obtained               %%
%%   - \department{...}   name of the graduate department           %%
%%   - \gradyear{...}     year of graduation                        %%
%%   - \author{...}       name of the author                        %%
%%   - \title{...}        title of the thesis                       %%
%%%%%%%%%%%%%%%%%%%%%%%%%%%%%%%%%%%%%%%%%%%%%%%%%%%%%%%%%%%%%%%%%%%%%%

%% ***   Change this example to appropriate values.   ***

%% Title stuff is in uh-thesis.cls

\degree{Master of Science} % Doctor of Philosophy
\department{Computer Science}
\gradyear{October 2017}
\author{\myAuthor}
\title{\myTitle}

%% ***   NOTE   ***
%% Put here all other formatting commands that belong in the preamble.


%% For example, to list only down to subsections in table of contents
%% (-1=part, 0=chapter, 1=section, 2=subsection, 3=subsubsection,
%%  4=paragraph, 5=subparagraph, 6=subsubparagraph).
%
\setcounter{tocdepth}{2}


%% ***   NOTE   ***
%% You should put all of your `\newcommand', `\newenvironment', and
%% `\newtheorem's (in other words, all the global definitions that
%% you will need throughout your thesis) in a separate file and use
%% "\input{filename}" to input it here.

\newboolean{AUTHORCOMMENTS}
\setboolean{AUTHORCOMMENTS}{true}

%!TEX Root=./paper.tex

\ifdefined\nocomments
\newcommand{\showindraft}[1]{}
\else
\newcommand{\showindraft}[1]{#1}
\fi


\newcommand{\dtx}[1]{}
% already defined in ut-thesis.cls
%\newcommand{\ignore}[1]{}


\newcommand{\instruct}[1]{\mathsf{#1}}
\newcommand{\caseof}[2]{\noindent \textbf{#1} #2:}
\newcommand*\step[1]{\tikz[baseline=(char.base)]{\node[shape=circle,draw,inner sep=0.8pt] (char) {#1};}}

\newcommand{\ie}{i.e.\ }
\newcommand{\eg}{e.g.,\ }
\newcommand{\etal}{et al.}
\newcommand{\vs}{vs.\ }
\newcommand{\etc}{etc.\ }

\newcommand{\toolname}{CHASER\xspace}

\newcommand{\fakeparagraph}[1]{\noindent\textbf{#1.}}

\newcolumntype{C}[1]{>{\centering\let\newline\\\arraybackslash\hspace{0pt}}m{#1}}
\newcolumntype{L}[1]{>{\raggedright\let\newline\\\arraybackslash\hspace{0pt}}m{#1}}
\newcolumntype{R}[1]{>{\raggedleft\let\newline\\\arraybackslash\hspace{0pt}}m{#1}}

\newcommand{\code}[1]{\texttt{#1}}

\newcommand{\SE}{SE}
\newcommand{\CSE}{CSE}

\algnewcommand\algorithmicforeach{\textbf{foreach}}
\algdef{S}[FOR]{ForEach}[1]{\algorithmicforeach\ #1\ \algorithmicdo}

\newcommand{\addr}{\mathit{addr}}
\newcommand{\as}{\mathit{as}}

\newcommand{\skipped}{\mathit{skipped}}
\newcommand{\inst}{\mathit{inst}}
\newcommand{\instate}{s_0}
\newcommand{\workstate}{s}
\newcommand{\runfunc}{\mathit{func}}
\newcommand{\skipFunctions}{\mathit{skipFunctions}}
\newcommand{\worklist}{\mathit{worklist}}

\newcommand{\snapshot}{\mathit{snapshot}}
\newcommand{\dependentState}{\mathit{dependentState}}
\newcommand{\slice}{\mathit{slice}}
\newcommand{\recoveryState}{\mathit{recoveryState}}
\newcommand{\isRecovery}{\mathit{isRecoveryState}}

\newcommand{\gc}{\mathit{guidingConstraints}}
\newcommand{\funclist}{\mathit{funclist}}
\newcommand{\owset}{\mathit{owset}}

\newcommand{\mtrue}{\mathit{true}}

\renewcommand*\Call[2]{\textproc{#1}(#2)}
% New definitions
\algrenewcommand\algorithmicindent{0.8em}
\algnewcommand\algorithmicswitch{\textbf{switch}}
\algnewcommand\algorithmiccase{\textbf{case}}
\algnewcommand\algorithmicassert{\texttt{assert}}
\algnewcommand\Assert[1]{\State \algorithmicassert(#1)}%
% New "environments"
\algdef{SE}[SWITCH]{Switch}{EndSwitch}[1]{\algorithmicswitch\ #1\ \algorithmicdo}{\algorithmicend\ \algorithmicswitch}%
\algdef{SE}[CASE]{Case}{EndCase}[1]{\algorithmiccase\ #1}{\algorithmicend\ \algorithmiccase}%
%\algtext*{EndSwitch}%
\algtext*{EndCase}%

%%% Local Variables:
%%% mode: latex
%%% TeX-master: "paper"
%%% End:


\floatsetup[figure]{objectset=centering}

%%%%%%%%%%%%      MAIN  DOCUMENT      %%%%%%%%%%%%

% Paper -> Thesis %
\let\subparagraph\paragraph
\let\paragraph\subsubsection
\let\subsubsection\subsection
\let\subsection\section
\let\section\chapter

%% This generates the index
\makeindex


\begin{document}


%% This sets the page style and numbering for preliminary sections.
\begin{preliminary}

%% This generates the title page from the information given above.
\maketitle

%% There should be NOTHING between the title page and abstract.

%% Anything placed between the abstract and table of contents will
%% appear on a separate page since the abstract ends with \newpage
%% and the table of contents starts with \clearpage.

\cleardoublepage

%% The `dedication' and `acknowledgements' sections do not create new
%% pages so if you want the two sections to appear on separate pages,
%% you should put an explicit \newpage between them.

%% This generates an "acknowledgements" section, if needed.
%% (uncomment to have it appear in the document)

%% This generates the abstract page, with the line spacing adjusted
%% according to SGS guidelines.

%!TEX root=./thesis.tex

\begin{abstract}
  Symbolic execution is a powerful program analysis technique that
  systematically explores multiple program paths. However, despite
  important technical advances, symbolic execution often struggles to
  reach deep parts of the code due to the well-known path explosion
  problem and constraint solving limitations.

  In this paper, we propose \textit{chopped symbolic execution}, a
  novel form of symbolic execution that allows users to specify
  uninteresting parts of the code to exclude during the analysis, thus
  only targeting the exploration to paths of importance. However, the
  excluded parts are not summarily ignored, as this may lead to both
  false positives and false negatives. Instead, they are executed
  lazily, when their effect may be observable by code under
  analysis. Chopped symbolic execution leverages various on-demand
  static analyses at runtime to automatically exclude code fragments
  while resolving their side effects, thus avoiding expensive manual
  annotations and imprecision.

  Our preliminary results show that the approach can effectively
  improve the effectiveness of symbolic execution in several different
  scenarios, including failure reproduction and test suite augmentation.
\end{abstract}


\cleardoublepage

\begin{acknowledgements}
TODO...

\end{acknowledgements}

\cleardoublepage

%% This generates the Table of Contents (on a separate page).
\tableofcontents

%% This generates the List of Tables (on a separate page), if needed.
%% (uncomment to have it appear in the document)
%\listoftables

%% This generates the List of Figures (on a separate page), if needed.
%% (uncomment to have it appear in the document)
%\listoffigures

%% End of the preliminary sections: reset page style and numbering.
\end{preliminary}

%%%%%%%%%%%%%%%%%%%%%%%%%%%%%%%%%%%%%%%%%%%%%%%%%%%%%%%%%%%%%%%%%%%%%%
%%  Put your Chapters here; the easiest way to do this is to keep   %%
%%  each chapter in a separate file and `\include' all the files    %%
%%  right here.  Note that each chapter file should start with the  %%
%%  line "\chapter{ChapterName}".  Note that using `\include'       %%
%%  instead of `\input' makes each chapter start on a new page.     %%
%%%%%%%%%%%%%%%%%%%%%%%%%%%%%%%%%%%%%%%%%%%%%%%%%%%%%%%%%%%%%%%%%%%%%%

%% ***   Include chapter files here.   ***
%%
%% Use include only at the top level - makes compilation faster and adds 
%% clearpage before and after every file.
%%


%\input{intro}
%\input{syscalls}
%\input{pl}
%\input{instrumented}
%\input{static-analysis}
%\input{contracts}
%\input{impl}
%\input{eval}
%% Extensions to the paper here
%\input{strings}
%\input{dynamic-calls}
%\input{inter-proc}
%\input{lifting}
%\input{se-extension}
%% Related work and conclusion
%%!TEX root=./thesis.tex

\chapter{Related Work}\label{chapter:related}

The research community has invested significant effort in addressing
the path explosion challenge in symbolic execution, and this paper
aligns with this line of work.

As we already mentioned in the introduction, the most common and often
most effective mechanism employed by symbolic executors are search
heuristics, whose goal is to guide program exploration to the most
promising paths in the program.  Popular heuristics include random
path exploration~\cite{klee}, generational search~\cite{sage} and
coverage-optimized search~\cite{exe,sen:concolicheuristics}, to name
just a few.  Unfortunately, search heuristics only partly alleviate
path explosion, and symbolic execution can still get stuck in
irrelevant parts of the code.

Another effective technique is to try to prune equivalent program
paths~\cite{exe:tacas,rwset2}.  For instance, if a path reaches a
program point with a set of constraints equivalent to those of a
previous path that reached that point, then the second path (and all
paths that it would have spawned) can be terminated.  This technique
is similar in spirit to our approach, but orthogonal, as it does
nothing to prevent the exploration of code irrelevant to the task at
hand.  Chopped symbolic execution can be combined with path pruning,
in order to prune both irrelevant paths, as well as those relevant
paths which are equivalent to other relevant paths.

Merging paths can also help alleviate path explosion.  Paths can be
merged either ahead-of-time~\cite{klee-fp,kleecl:tse} or at
runtime~\cite{merging:pldi12,multise:fse15}. A particular type of path
merging are function summaries, in which paths within a function are
merged into a summary that can be reused on subsequent
invocations~\cite{godefroid:popl,godefroid:tacas}. Path merging can
lead to exponential reduction in the number of paths explored, but the
cost is often off-loaded to the constraint solver, which has to deal
with significantly harder constraints.  Again, chopped symbolic
execution could be combined with path merging, in order to get the
benefit of both.

Chopped symbolic execution makes use of program slicing in order to
explore only the relevant parts of code through the skipped functions.
Program slicing has been explored in symbolic execution before, \eg in
the context of patch testing~\cite{babic11}.


%%% Local Variables:
%%% mode: latex
%%% TeX-master: "paper"
%%% End:

%\input{conc}
%\appendix
%\input{contracts-verbatim}
%\input{notes-at-the-end}

%% This adds a line for the Bibliography in the Table of Contents.
\addcontentsline{toc}{chapter}{Bibliography}
%% ***   Set the bibliography style.   ***
%% (change according to your preference)
\bibliographystyle{plain}
%\bibliographystyle{abbrv}
%% ***   Set the bibliography file.   ***
%% ("thesis.bib" by default; change if needed)
%\bibliography{bib,concurrent,pldi09/ds}
% \bibliography{biblio-string-long,biblio,thesis}
\bibliography{bib_files/cadar-macros,bib_files/cadar,bib_files/cadar-crossrefs,bib_files/cse}

%%\input{etl/states/proofs.tex}
%\input{appendices/logicsummary.tex}
%\chapter{Additional Proofs}\label{Ch:Proofs}
%\input{appendices/etl_traces_proofs.tex}
%\input{appendices/etl_states_proofs.tex}
%\includefrom*{appendices/}{etl_states_appendices}


%\includefrom*{qnf/}{qnf_installation}


%% ***   NOTE   ***
%% If you don't use bibliography files, comment out the previous line
%% and use \begin{thebibliography}...\end{thebibliography}.  (In that
%% case, you should probably put the bibliography in a separate file
%% and `\include' or `\input' it here).

%\printindex OS

%\appendix
%\include{appendix}


\end{document}

%%%%%%%%%%%%%%%%%%%%%%%%%%%%%%%%%%%%%%%%%%%%%%%%%%%%%%%%%%%%%%%%%%%%%%
%%  End of UT-THESIS.TEX
%%%%%%%%%%%%%%%%%%%%%%%%%%%%%%%%%%%%%%%%%%%%%%%%%%%%%%%%%%%%%%%%%%%%%%
