%!TEX root=./paper.tex

\chapter{Implementation}\label{chapter:implementation}

We implemented chopped symbolic execution into \toolname, an extension
to the KLEE symbolic execution
engine~\cite{klee}.\footnote{http://url-omitted-for-double-blind-revision}
We forked KLEE from commit~\code{b2f93ff}. A user can run \toolname by
specifying the list of functions to skip along with specific call
sites via command-line switches.

\section{Static Analysis}
Our tool combines static analysis---in
particular mod-ref analysis and slicing---along with symbolic
execution. Since KLEE operates on LLVM bitcode, we rely on libraries
that statically analyze LLVM bitcode. In particular, we implemented a
library for static analysis that exposes APIs to KLEE, so new or
better static analyses can be integrated in \toolname with ease.

We compute \textit{mod-ref analysis} by using the pointer analysis
provided by SVF~\cite{sui2016svf}. In particular, we rely on a
flow-insensitive and context-insensitive pointer analysis based on the
Andersen algorithm~\cite{andersen:pointeranalysis}. We compute static
backward slicing using the DG static slicer~\cite{dg}. We modified the
slicer to be able to generate slices of arbitrary functions and not
only of the entry point of the program. Note that static slicing is
computed on-demand, only when a recovery is required. The same slice
may be reused for multiple recoveries, so each slice is computed only
once.

\section{State Exploration}
\toolname also implements its own state
prioritization algorithm, \textit{chopped-aware search heuristic},
which favors the execution of normal states over recovery ones as
discussed in \S~\ref{Se:Search}. The algorithm is implemented as a
\textit{state searcher}, an abstraction used by KLEE to encapsulate
the state exploration logic. This implementation strategy also allows
us to be search-heuristic agnostic, meaning that our prioritization
strategy is compatible and can be combined with all present and future
search heuristics of KLEE.

\section{Limitations}
The current main limitation of \toolname is
the handling of symbolic addresses. Handling a symbolic address in
\toolname is not trivial, as it may refer to multiple allocation sites
and in turn lead to the recovery of several different skipped
functions. Moreover, symbolic addresses also increase the complexity
of interpreting load instructions. In case of a load instruction,
\toolname needs a concrete address to update in the dependent state
the addresses written during recovery, as discussed in \Cref{chapter:design}.

Chopped symbolic execution currently focuses on skipping functions.
However, the approach is more generic: In theory, we could skip any
arbitrary code portion that preserves the control-flow of the
program. We are currently working on relaxing the implementation of
\toolname to support such feature, and on designing an appropriate API
for specifying such arbitrary code portions.

%%% Local Variables:
%%% mode: latex
%%% TeX-master: "paper"
%%% End:
