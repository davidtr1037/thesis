%!TEX root=./paper.tex

\section{Main Algorithm}\label{section:main-algorithm}

\Cref{fig:chopped-symbexe} presents the key steps in chopped
symbolic execution, which we gradually explain. The algorithm operates
on a simple imperative C-like heap-manipulating language with
assignments, assertions, conditional jumps, dynamic memory allocation
and reclamation, and function calls with call-by-value parameter
passing.
Note that in practice, our algorithm operates on LLVM bitcode~\cite{llvm}.
The function parameters can be of any type: primitives, structs, poitner, \etc
Without loss of generality, we assume that functions do not have a return value.
A function with a return value can always be rewritten with an additional parameter 
that points to the memory location of the return value.
For example, the function in figure~\ref{fig:ret2void-original-f}
will be rewritten to the function in figure~\ref{fig:ret2void-rewritten-f}.
\begin{figure*}[t]
  \centering
  \subfloat[]{
    \lstinputlisting[linewidth=.4\textwidth,numbers=none]{code/ret2void-1.c}
    \label{fig:ret2void-original-f}
  }
  
  \subfloat[]{
    \lstinputlisting[linewidth=.4\textwidth,numbers=none]{code/ret2void-2.c}
    \label{fig:ret2void-rewritten-f}
  }

  \caption{Reduction to a void-returning function}
  \label{fig:simple}
\end{figure*}

To simplify the explanation, we now
assume that we may skip at most one function invocation at every
explored path, and discuss the general case in
\S\ref{Se:SkipMultipleFuncs}. For the same reason, we also assume that
the program does not dynamically allocate memory, and discuss the
handling of $\code{malloc}$ and $\code{free}$ in \S\ref{Se:Malloc}.

Chopped symbolic execution begins by invoking procedure $\textsc{cse}$
with an \emph{initial symbolic state} ($\instate$) and a set
containing the names of the functions that the user wishes to skip
($\skipFunctions$). We expect a symbolic state $\workstate$ to encode,
among other properties, the next instruction to be executed, the activation record stack,
and a symbolic description of the program \emph{address space}. For example, the
chopped symbolic execution described in \Cref{chapter:overview} begins
with $\instate$ in which the stack contains only the activation record
of $\code{main}$, with the program counter at
line~\ref{line:main_enter}, an empty address space, and
$\skipFunctions = \{ \code{f} \}$.

The pool of execution states is mainted using the $\worklist$,
which is initialized at the beginning with $\instate$ (\cref{alg:seed-worklist}).
Then, a standard worklist-based algorithm starts executing until either the
worklist is empty (\cref{alg:iterate-worklist}), or the algorithm
exhausts the time budget (elided). As usual, the algorithm pops the
symbolic state $\workstate$ to explore out of the worklist
(\cref{alg:pop-worklist}).
Unconventionally, however, we assume that
the selected state is not suspended, since a suspended state is ignored by the execution engine
until it's corresponding recovery state terminates
and the value of the depended load is recovered (see \S\ref{chapter:overview}).
At this point, we defer the discussion regarding the search heuristics to \S\ref{Se:Search},
and assume, for now, that any unsuspended state can be returned.
The next step of the algorithm depends on the instruction type (\cref{alg:switch}).

\subsection{\text{Handling \textit{Call} Instructions}}
A $\instruct{Call}$ instruction is handled at lines~\ref{alg:casecall-begin}--\ref{alg:casecall-end} as illustrated by step
\step{1} in \Cref{fig:simple} (see \S\ref{chapter:overview}):
First, the algorithm determines the name $f$ of the invoked function
(\cref{alg:call-find-target}). Then, if $f$ is one of the skipped
functions, the algorithm creates a snapshot of the current state
$\workstate$ and advances the program counter in $\workstate$
without executing the skipped function (\cref{alg:take-snapshot}).
The newly created $\snapshot$ state is added to the end of its list 
of \emph{skipped invocations} (\cref{alg:record-snapshot}).
A skipped invocation is represented
as a triple $(f,\snapshot,\owset)$ composed of the name of a skipped
function $f$, a $snapshot$ of the symbolic state at the time $f$ was
skipped, and a set of the addresses $\owset$ subsequently overwritten
(initially empty).
If at some point we will need to recover some of the side effects
of the skipped function, the recovery state will be created
by cloning the $\snapshot$ state.

Conversely, if $f$ is not be skipped, the algorithm handles its
invocation as usual in symbolic execution.  For brevity, we elide the
standard handling of commands by symbolic execution.

\subsection{\text{Handling \textit{Load} Instructions}}
A $\instruct{Call}$ instruction is handled
at lines {(lines~\ref{alg:caseload-begin}--\ref{alg:caseload-end})}.
The algorithm utilizes
$\Call{mayMod}{\workstate,\workstate.\skipped,\addr}$, as shown in
\Cref{fig:aux-func-may-mod} and explained in
\S\ref{Se:IdentifyingLoads}, to determine whether the address from
which a value is read ($\addr$) might have been modified by one of the
skipped functions on the path followed by the current state
$\workstate$.
If so, the algorithm creates a recovery state by
calling $\Call{recover}{\workstate,\addr}$.  Otherwise, the
$\instruct{Load}$ instruction is handled as usual in symbolic
execution (\cref{alg:load-normal}).

Procedure \textsc{createRecoveryState} is shown in
\ref{fig:aux-func-recS}. The function handles $\instruct{Load}$
instructions as illustrated by step \step{2}{} in \Cref{fig:simple}
(see \S\ref{chapter:overview}): It iterates over the list of skipped
functions (\Cref{alg:recover-foreach}), and uses
$\Call{mayMod}{\workstate, f, \addr}$ to determine which of the skipped
functions $f$ might have modified the dependent address. Once it finds
such a function (\cref{alg:recover-if-found}), the current state
becomes a dependent state (\cref{alg:recover-gen-depS}) and it is
immediately suspended (\cref{alg:recover-suspend}). The procedure then
generates the corresponding \emph{recovery state} $\recoveryState$ by
forking $\snapshot$ and by augmenting its path condition with the
guiding constraints $\gc$ (\cref{alg:recover-get-gc}), \ie the path
conditions accumulated in $\workstate$ since the snapshot state was
created (\cref{alg:recover-gen-recS,alg:recover-set-is-recS}). The
algorithm then invokes a static program slicer to remove from the
skipped function $f$ instructions which cannot affect the address of
the dependent load (\cref{alg:recover-slice}); records that
$\recoveryState$ was spawned to determine the value written in address
$\addr$ of $\dependentState$ (\cref{alg:recover-record-dep}); and
pushes the two states into the worklist
(\cref{alg:push-worklist-recovery}).

\subsection{\text{Handling \textit{Branch} Instructions}}
%{(lines~\ref{alg:casebranch-begin}--\ref{alg:casebranch-end})}

The algorithm checks whether the current state $\workstate$ is a
recovery state. If so, then the $\instruct{Branch}$ instruction is
handled as illustrated by steps \step{4}{} and \step{5}{} in
\Cref{fig:simple} (see \S\ref{chapter:overview}): It first extracts from
the worklist the (suspended) dependent state $\dependentState$, which
spawned $\workstate$ as a recovery state
(\cref{alg:extract-dependent}).  It then determines the branch
condition $\varphi$ (\cref{alg:branch-get-cond}); forks both the
current (recovery) state $\workstate$ and the dependent state
$\dependentState$, and adds $\varphi$ to their path condition
(lines~\ref{alg:branch-fork-true1}--\ref{alg:branch-fork-true2}).
After the fork, it checks whether the resulting states are feasible,
\ie their path conditions are satisfiable
(\cref{alg:branch-feasible-true}), and if so, adds them to the
worklist. If either one is not feasible, the newly forked recovery and
dependent states are simultaneously discarded
(\cref{alg:push-worklist-true-branch}).
Lines~\ref{alg:branch-fork-true1}--\ref{alg:push-worklist-true-branch}
act similarly to
lines~\ref{alg:branch-fork-false1}--\ref{alg:push-worklist-false-branch},
except that we use the negation of the branch condition
$\neg\varphi$. If the current state $\workstate$ is not a recovery
state, then the $\instruct{Branch}$ instruction is handled as usual in
symbolic execution (\cref{alg:branch-normal}).

\subsection{\text{Handling \textit{Store} Instructions}}
%{(lines~\ref{alg:casestore-begin}--\ref{alg:casestore-end})}

The algorithm executes the $\instruct{Store}$ instruction on the
current state in two steps. First, it updates the set of overwritten
addresses of the skipped function to record that a value was stored in
$\addr$ after the skip, and thus any value they may write is no longer
relevant (\cref{alg:record-overwrite}). Lastly, the symbolic state
$\workstate$ is updated as usual in symbolic execution
(\cref{alg:store-normal}). If $\workstate$ is a recovery state, then
the algorithm invokes
\Call{udpateDependentStates}{$\workstate, \addr$} (code elided) to
updated the dependent state (as illustrated by step \step{6}{} in
\Cref{fig:simple}, see \S\ref{chapter:overview}). In addition, it also
updates the set of overwritten addresses in the list of skipped
invocations of the dependent state. Note that
\Call{udpateDependentStates}{$\dependentState, \inst$} updates, but
does not resume, the dependent state.

\subsection{\text{Handling \textit{Exit} Instructions}}
%{(lines~\ref{alg:caseexit-begin}--\ref{alg:caseexit-end})}

The algorithm checks whether the current state $\workstate$ is a
recovery state. If $\workstate$ is a recovery state \emph{and} the
$\instruct{Exit}$ instruction is invoked inside the skipped function,
then the recovery is terminated and the instruction is handled as
illustrated by step \step{7}{} in \Cref{fig:simple} (see
\S\ref{chapter:overview}): Specifically, the recovery state itself if
discarded (\cref{alg:exit-terminate-recovery}) and the dependent state
is resumed \cref{alg:exit-resume}. Otherwise, the $\instruct{Exit}$
instruction is handled as usual in symbolic execution
(\cref{alg:exit-normal}).

%!TEX Root=./paper.tex

\begin{algorithm} % [tbp]
\caption{Chopped symbolic execution (simplified).\label{fig:chopped-symbexe-recover}\label{fig:chopped-symbexe} }
\begin{algorithmic}[1]
  \Function{cse}{$\instate$, $\skipFunctions$}
  \State $\worklist \gets \worklist \cup \{ \instate\}$  \label{alg:seed-worklist}
  \While{$\worklist \neq \emptyset$}                     \label{alg:iterate-worklist}
    \State $\workstate \gets  \Call{selectUnsuspended}{\worklist}$ \label{alg:pop-worklist}
    \State $\inst \gets  \Call{instruction}{\workstate}$ \label{alg:get-switch-inst}

    \Switch{$\inst$} \label{alg:switch}
    \Case{Call} \label{alg:casecall-begin}
      \Call{handleCall}{$\workstate, \inst$}
    \EndCase \label{alg:casecall-end}
    \Case{Load} \label{alg:caseload-begin}
      \Call{handleLoad}{$\workstate, \inst$}
    \EndCase \label{alg:caseload-end}
    \Case{Branch} \label{alg:casebranch-begin}
      \Call{handleBranch}{$\workstate, \inst$}
    \EndCase \label{alg:casebranch-end}
    \Case{Store} \label{alg:casestore-begin}
      \Call{handleStore}{$\workstate, \inst$}
    \EndCase \label{alg:casestore-end}
    \Case{Exit} \label{alg:caseexit-begin}
      \Call{handleExit}{$\workstate, \inst$}
    \EndCase \label{alg:caseexit-end}
    \EndSwitch
  \EndWhile
  \EndFunction
\end{algorithmic}
\end{algorithm}

\begin{algorithm}
  \caption{Auxiliary procedure: \textsc{handleCall}
  \label{fig:aux-func-recS}}
\begin{algorithmic}[1]

\Function{handleCall}{$\workstate, \inst$}
\State $f \gets  \Call{targetFunction}{\workstate}$ \label{alg:call-find-target}
\If{$f \in \skipFunctions$}
  \State $\snapshot \gets \Call{createSnapshotAndSkip}{\workstate}$ \label{alg:take-snapshot}
  \State $\workstate.\skipped \gets \workstate.\skipped + (f,\snapshot,\emptyset)$ \label{alg:record-snapshot}
\Else
  \State $\Call{executeCall}{\workstate}$ \label{alg:call-normal}
\EndIf
\EndFunction
\end{algorithmic}
\end{algorithm}

\begin{algorithm}
  \caption{Auxiliary procedure: \textsc{handleLoad}
  \label{fig:aux-func-recS}}
\begin{algorithmic}[1]

\Function{handleLoad}{$\workstate, \inst$}
\State $\addr \gets \Call{loadAddress}{\workstate}$
\If{$\Call{mayMod}{\workstate,\workstate.\skipped,\addr}$}%.\skipped}$}
  %\TODO: Explain that this is a simplification and recovery states are maintained in a pool
  \State $\Call{createRecoveryState}{\workstate,\addr}$ \label{alg:call-recover}
\Else
  \State \Call{executeLoad}{$\workstate, \inst$}  \label{alg:load-normal}
\EndIf
\EndFunction
\end{algorithmic}
\end{algorithm}

\begin{algorithm}
  \caption{Auxiliary procedure: \textsc{handleBranch}
  \label{fig:aux-func-recS}}
\begin{algorithmic}[1]

\Function{handleBranch}{$\workstate, \inst$}
\If{$\workstate.\isRecovery$} \label{alg:branch-check-recovery}
  \State $\varphi = \Call{condition}{\inst}$  \label{alg:branch-get-cond}
  \State $\dependentState \gets  \Call{getDependent}{\workstate}$ \label{alg:extract-dependent}
  \State $\workstate'' \gets \Call{fork}{\workstate,\varphi }$ \label{alg:branch-fork-true1}
  \State $\dependentState'' \gets \Call{fork}{\dependentState,\varphi }$ \label{alg:branch-fork-true2}
  \If{$\Call{feasible}{\workstate''} \land \Call{feasible}{\dependentState''}$} \label{alg:branch-feasible-true}
    \State $\worklist  \gets  \worklist  \cup \{\workstate'', \dependentState''\}$ \label{alg:push-worklist-true-branch}
  \EndIf
  \State $\workstate' \gets \Call{fork}{\workstate, \neg\varphi}$ \label{alg:branch-fork-false1}
  \State $\dependentState' \gets \Call{fork}{\dependentState, \neg\varphi}$ \label{alg:branch-fork-false2}
  \If{$\Call{feasible}{\workstate'} \land \Call{feasible}{\dependentState'}$}  \label{alg:branch-feasible-false}
    \State $\worklist  \gets  \worklist  \cup \{\workstate'', \dependentState''\}$ \label{alg:push-worklist-false-branch}
  \EndIf
\Else
  \State \Call{executeBranch}{$\workstate$}  \label{alg:branch-normal}
\EndIf
\EndFunction
\end{algorithmic}
\end{algorithm}

\begin{algorithm}
  \caption{Auxiliary procedure: \textsc{handleStore}
  \label{fig:aux-func-recS}}
\begin{algorithmic}[1]

\Function{handleStore}{$\workstate, \inst$}
\State $\addr \gets \Call{storeAddress}{\workstate}$
\State $\Call{recordOverwrite}{\workstate,\addr}$     \label{alg:record-overwrite}
\State $\Call{executeStore}{\workstate,\addr}$     \label{alg:store-normal}
\If{$\workstate.\isRecovery$} \label{alg:store-check-recovery}
  \State \Call{updateDependentState}{$\workstate, \addr$} \label{alg:store-udpate-dep-states}
  \ignore{
    \If{$\addr = $\Call{dependentLoad}{$\workstate$}}
      \State $\dependentState \gets  \Call{getDependent}{\workstate}$
      \label{alg:store-extract-dependent}
      \State \Call{executeStore}{$\dependentState, \inst$}
    \EndIf
  }
\EndIf
\EndFunction
\end{algorithmic}
\end{algorithm}

\begin{algorithm}
  \caption{Auxiliary procedure: \textsc{handleExit}
  \label{fig:aux-func-recS}}
\begin{algorithmic}[1]

\Function{handleExit}{$\workstate, \inst$}
\If{$\workstate.\isRecovery \land \Call{retInSkip}{\workstate}$} \label{alg:exit-check-recovery}
  \State \Call{terminate}{$\recoveryState$} \label{alg:exit-terminate-recovery}
  \State $\dependentState \gets  \Call{getDependent}{\workstate}$ \label{alg:exit-extract-dependent}
  \State \Call{resume}{$\dependentState$} \label{alg:exit-resume}
  \State $\worklist \gets  \worklist \cup \{\dependentState\}$
\Else
  \State \Call{executeExit}{$\workstate$}  \label{alg:exit-normal}
  \label{alg:push-worklist-exit}
\EndIf
\EndFunction
\end{algorithmic}
\end{algorithm}

\begin{algorithm}
  \caption{Auxiliary procedure: \textsc{createRecoveryState}
  \label{fig:aux-func-recS}}
\begin{algorithmic}[1]

\Function{createRecoveryState}{$\workstate, \addr$}
  \ForEach{$(f,\snapshot,\owset) \in \workstate.\skipped$} \label{alg:recover-foreach}
    \If{$\Call{mayMod}{\workstate, (f,\snapshot,\owset) ,\addr}$} \label{alg:recover-if-found}
       \State $\dependentState \gets \workstate$ \label{alg:recover-gen-depS}
       \State $\Call{suspend}{\dependentState}$ \label{alg:recover-suspend}
       \State $\gc \gets  \Call{getGuidingConstraints}{\dependentState}$ \label{alg:recover-get-gc}
       \State $\recoveryState \gets \Call{fork}{\snapshot,\gc}$ \label{alg:recover-gen-recS}
       \State $\recoveryState.\isRecovery \gets \mtrue$ \label{alg:recover-set-is-recS}
       % TODO: fix the slice call...
       \State $\Call{slice}{\recoveryState,\addr}$ \label{alg:recover-slice}
       \State $\recoveryState.dependentState = \dependentState$ \label{alg:set-dependent-state}
       \State $\recoveryState.loadAddr = \addr$ \label{alg:set-load-addr}
       \State $\worklist \gets \worklist \cup \{\recoveryState\}$
       \label{alg:push-worklist-recovery}
    \EndIf
  \EndFor
\EndFunction
\end{algorithmic}
\end{algorithm}

\begin{algorithm}
  \caption{Auxiliary procedure: \textsc{mayMod}().
  \label{fig:aux-func-may-mod}}
\begin{algorithmic}[1]


\Function{mayMod}{$\workstate, \funclist, \addr$}
  \ForEach{$(f,\snapshot,\owset) \in \funclist$}
    \If{$\Call{allocSite}{\workstate,\addr} \in \Call{modSet}{f}$} \label{alg:maymod-static}
      \If{$\addr\not\in\owset$}  \label{alg:maymod-dynamic}
         \State \Return{$\mathit{true}$}
       \EndIf
    \EndIf
  \EndFor
  \State \Return{$\mathit{false}$}
\EndFunction

\end{algorithmic}
\end{algorithm}


%%% Local Variables:
%%% mode: latex
%%% TeX-master: "paper"
%%% End:


%%% Local Variables:
%%% mode: latex
%%% TeX-master: "paper"
%%% End:
