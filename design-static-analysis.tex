%!TEX root=./thesis.tex

\section{Static Inference of Function Side-Effects}
\label{section:static-analysis}

\subsection{Pointer Analysis}
We perform a whole program flow-insensitive, context-insensitive, and field-sensitive
points-to analysis~\cite{Hind:Paste2001, Smaragdakis:FTPL2015} which
determines, in a conservative way, the memory location each pointer
variable may point-to. In this analysis, memory locations are
conservatively abstracted using their \emph{allocation sites}: Every
definition of a local or a global variable is considered to be an
allocation site, as well as every program point in which memory is
allocated.
For example, in the program from figure~\ref{fig:static-analysis},
all the memory objects which are allocated at line~\ref{line:object_as}
are represnted by the same allocation site $\mathit{AS_1}$ at line~\ref{line:object_as}
and the locally allocated $\code{objects}$ array is represneted
by a single allocation site $\mathit{AS_2}$ at line~\ref{line:objects_as}.
We then say that $\code{o}$ may point to allocation
site $\mathit{AS_1}$, and $\code{objects}$ may point to allocation site $\mathit{AS_2}$.
The nodes of the points-to graph of a
program are the variable names and allocation sites, and its edges
represent \emph{points-to relations}:
an edge from node $v$ to $w$ means that the memory location
represented by $v$ may hold a pointer to $w$.
The points-to graph, which is computed once for every program,
conservatively represents all the possible points-to relation in any
possible program execution. 

\begin{figure}
  \captionsetup[subfloat]{labelformat=empty}
  \subfloat[]{
    \lstinputlisting{code/static-analysis.c}
  }
  \caption{Allocation sites in pointer analysis}
  \label{fig:static-analysis}
\end{figure}

\subsubsection{Structs}
Our pointer analysis treats structs in a field-sensitive way.
In figure~\ref{fig:static-analysis-structs},
the analysis can distinguish between the memory location of $\code{p->x}$ at line \ref{line:pa_px_store}
and the memory location of $\code{p->y}$ at line \ref{line:pa_py_store}.
In field-sensitive objects, field offsets are used in order to distinguish between different fields.
In this case, the allocation site pointed by $\code{p->x}$ is represented by
the local variable at line~\ref{line:pa_p_as} and the \textit{offset} 0.
Similarly, the allocation site pointed by $\code{p->y}$ is represented by
the local variable at line~\ref{line:pa_p_as} and the \textit{offset} 4.

\begin{figure}
  \captionsetup[subfloat]{labelformat=empty}
  \subfloat[]{
    \lstinputlisting[linewidth=.4\textwidth]{code/static-analysis-structs.c}
  }
  \caption{Allocation sites of structs}
  \label{fig:static-analysis-structs}
\end{figure}

\subsubsection{Arrays}
Unsurprisingly, arrays are treated in a field-insensitive way (or index-insensitive way).
In figure~\ref{fig:static-analysis-arrays},
the analysis treats $\code{tokens[0]}$ and $\code{tokens[1]}$ as the same memory location.
As a result, the points-to set of the memory location at line~\ref{line:pa_token}
includes the allocation sites of both the string \textit{"a"} and the string \textit{"b"}.

\begin{figure}
  \captionsetup[subfloat]{labelformat=empty}
  \subfloat[]{
    \lstinputlisting[linewidth=.4\textwidth]{code/static-analysis-arrays.c}
  }
  \caption{Allocation sites of structs}
  \label{fig:static-analysis-arrays}
\end{figure}

\subsection{Mod-Ref Analysis}
Using the points-to graph, we use a standard \emph{may-mod} analysis (see, e.g.,~\cite{dragon-book}),
in which we find the side effects of every function $f$,
\ie the set of possible locations, represented by their allocation sites,
that the function itself or any function that it may (transitively) invoke, may modify.

In addition to computing the side effects,
we compute for each \textit{load} instruction whose operand may be modified by a skipped function,
the set of allocation sites which are pointed by this operand and may be modified by a skipped function.
We will discuss it in more details in section~\ref{section:main-algorithm}.


%%% Local Variables:
%%% mode: latex
%%% TeX-master: "thesis"
%%% End:
