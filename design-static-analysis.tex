%!TEX root=./paper.tex

\section{Static Inference of Function Side-Effects}
\label{Se:MayMod}
\label{Se:IdentifyingLoads}

The auxiliary function $\Call{mayMod}{ \workstate, \funclist, \addr }$
receives as parameters a symbolic state $\workstate$, a list of
skipped invocations $\funclist$, and an address $\addr$ which is the
target of a $\instruct{Load}$ instruction, and determines whether one
of the skipped function in $\funclist$ may store a value in $\addr$.
The function makes this decision using a \emph{points-to} graph
computed by a preliminary stage of pointer
analysis~\cite{Hind:Paste2001, Smaragdakis:FTPL2015}.

More specifically, we perform a whole program flow-insensitive,
context-insensitive, and field-sensitive points-to analysis which
determines, in a conservative way, the memory location each pointer
variable may point-to. In this analysis, memory locations are
conservatively abstracted using their \emph{allocation sites}: Every
definition of a local or a global variable is considered to be an
allocation site, as well as every program point in which memory is
allocated. For example, if the program contains
$\code{while (..) do L: p=malloc(4)}$ then we represent all the memory
locations allocated in $L$ by a single allocation site
$\mathit{AS}_L$.  We then say that $\code{p}$ may point to allocation
site $\mathit{AS}_L$, and if the program contains $\code{p=q}$, we say
the same about $\code{q}$. The nodes of the points-to graph of a
program are the variable names and allocation sites, and its edges
represent \emph{points-to relations}: An edge from node $v$ to $w$
means that the memory location represented by $v$ may hold a pointer
to $w$.

The points-to graph, which is computed once for every program,
conservatively represents all the possible points-to relation in any
possible program execution. Using the points-to graph, we use a
standard \emph{may-mod} analysis (see, e.g.,~\cite{dragon-book}), in
which we find the side effects of every function $f$, \ie the set of
possible locations, represented by their allocation sites, that the
function itself or any function that it may (transitively) invoke, may
modify.

During the chopped symbolic execution, we instrument the symbolic
state to record the allocation site of every memory location. This
instrumentation, together with the program points-to graph, allows
\textsc{mayMod} to determine whether a skipped function may write to a
given address. Recall that the pointer analysis is flow-insensitive,
and thus it might record that a skipped function might modify a
location which is updated later on in the symbolic execution.  More
specifically, a $\instruct{load}$ instruction from address $\addr$ is
\emph{dependent on an invocation of a skipped function} if and only
if: (1)~$\addr$ is among the locations that \textit{may} be modified
by the skipped function (according to the may-mod analysis), and
(2)~no stores to that location happened between the skipped invocation
function and the load. In particular, when the second condition
doesn't hold, no recovery is needed as the stores performed by the
skipped function are irrelevant. $\Call{mayMod}$ utilizes the
information gathered during the symbolic execution regarding
overwritten locations to refine on-the-fly the detection of the
\emph{relevant} side effects of skipped functions.

%%% Local Variables:
%%% mode: latex
%%% TeX-master: "paper"
%%% End:
