%!TEX root=./thesis.tex

\begin{abstract}
  Symbolic execution is a powerful program analysis technique that
  systematically explores multiple program paths. However, despite
  important technical advances, symbolic execution often struggles to
  reach deep parts of the code due to the well-known path explosion
  problem and constraint solving limitations.

  In this thesis, we propose \textit{chopped symbolic execution}, a
  novel form of symbolic execution that allows users to specify
  uninteresting parts of the code to exclude during the analysis, thus
  only targeting the exploration to paths of importance. However, the
  excluded parts are not summarily ignored, as this may lead to both
  false positives and false negatives. Instead, they are executed
  lazily, when their effect may be observable by code under
  analysis. Chopped symbolic execution leverages various on-demand
  static analyses at runtime to automatically exclude code fragments
  while resolving their side effects, thus avoiding expensive manual
  annotations and imprecision.

  Our preliminary results show that the approach can effectively
  improve the effectiveness of symbolic execution in several different
  scenarios, including failure reproduction and test suite augmentation.
\end{abstract}
