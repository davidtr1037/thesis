%!TEX root=./paper.tex

\subsection{Static Inference of Function Side-Effects}
\label{Se:MayMod}
\label{Se:IdentifyingLoads}

The auxiliary function $\Call{mayMod}{ \workstate, \funclist, \addr }$
receives as parameters a symbolic state $\workstate$, a list of
skipped invocations $\funclist$, and an address $\addr$ which is the
target of a $\instruct{Load}$ instruction, and determines whether one
of the skipped function in $\funclist$ may store a value in $\addr$.
The function makes this decision using a \emph{points-to} graph
computed by a preliminary stage of pointer
analysis~\cite{Hind:Paste2001, Smaragdakis:FTPL2015}.

More specifically, we perform a whole program flow-insensitive,
context-insensitive, and field-sensitive points-to analysis which
determines, in a conservative way, the memory location each pointer
variable may point-to. In this analysis, memory locations are
conservatively abstracted using their \emph{allocation sites}: Every
definition of a local or a global variable is considered to be an
allocation site, as well as every program point in which memory is
allocated. For example, if the program contains
$\code{while (..) do L: p=malloc(4)}$ then we represent all the memory
locations allocated in $L$ by a single allocation site
$\mathit{AS}_L$.  We then say that $\code{p}$ may point to allocation
site $\mathit{AS}_L$, and if the program contains $\code{p=q}$, we say
the same about $\code{q}$. The nodes of the points-to graph of a
program are the variable names and allocation sites, and its edges
represent \emph{points-to relations}: An edge from node $v$ to $w$
means that the memory location represented by $v$ may hold a pointer
to $w$.

The points-to graph, which is computed once for every program,
conservatively represents all the possible points-to relation in any
possible program execution. Using the points-to graph, we use a
standard \emph{may-mod} analysis (see, e.g.,~\cite{dragon-book}), in
which we find the side effects of every function $f$, \ie the set of
possible locations, represented by their allocation sites, that the
function itself or any function that it may (transitively) invoke, may
modify.

During the chopped symbolic execution, we instrument the symbolic
state to record the allocation site of every memory location. This
instrumentation, together with the program points-to graph, allows
\textsc{mayMod} to determine whether a skipped function may write to a
given address. Recall that the pointer analysis is flow-insensitive,
and thus it might record that a skipped function might modify a
location which is updated later on in the symbolic execution.  More
specifically, a $\instruct{load}$ instruction from address $\addr$ is
\emph{dependent on an invocation of a skipped function} if and only
if: (1)~$\addr$ is among the locations that \textit{may} be modified
by the skipped function (according to the may-mod analysis), and
(2)~no stores to that location happened between the skipped invocation
function and the load. In particular, when the second condition
doesn't hold, no recovery is needed as the stores performed by the
skipped function are irrelevant. $\Call{mayMod}$ utilizes the
information gathered during the symbolic execution regarding
overwritten locations to refine on-the-fly the detection of the
\emph{relevant} side effects of skipped functions.

\subsection{Multiple Recovery States}
\label{Se:MultiRecovery}
In some cases, we need to create several recovery states during a
single chopped symbolic execution.

For example, consider the following code fragment which modifies the
body of the \code{main()} function in Figure~\ref{fig:simple}:
\[
\code{f(\&p,k); if (p.x) \{ p.z++; \} if (p.y) \{ p.z--; \}}
\]
If we wish to skip the invocation to $\code{f}()$ then a recovery
state and a dependent state are created every time the condition of
the branch needs to be evaluated. Note that the second dependent state
is produced from the first dependent one and that the resumed state
encapsulates the changes made by the first recovery state. Assume that
these changes involve a modification of the value of $\code{p.x}$
inside the $k>0$ branch at line~\ref{line:px_write}. If the symbolic
execution of the second recovery state goes through the path in which
$\code{p.y}$ is updated ($k\le0$), the induced combined execution
would be infeasible. To avoid this undesirable situation, when a
recovery state terminates, it adds the new constraints accumulated in
its path condition to the \textit{guiding constraints} of its dependent state. The added
constraints are then used in subsequent recovery states. In our
example in Figure~\ref{fig:simple}, the constraint $k > 0$ is
propagated from the first recovery state to the first dependent state,
thus ensuring that the symbolic execution of the second recovery state
does not follow an infeasible path.

%!TEX root=./paper.tex

\subsection{Handling Multiple Skipped Functions}
\label{Se:SkipMultipleFuncs}
So far, we have assumed that every symbolic state has a single skipped
invocation. When we have multiple invocations that can be skipped, we
then need to decide which functions to use for recovery and in which
order, in the case multiple invocations modify the dependent load
address $\addr$. We solve this issue by executing the skipped
invocations according to their order along the path, thus ensuring
that the value stored in $\addr$ at the end of the recovery process is
indeed the last value written there along the chopped path.

Another issue that we need to address to support multiple skipped
functions is that a skipped invocation might depend on the side
effects of an earlier skipped functions. When this happens, we apply
our recovery approach in a recursive manner, and treat the current
recovery state as a \emph{dependent} state. For example, consider the
code in Figure~\ref{fig:multiple-skipped-functions}. When the
execution reaches the dependent load at line~13, we create a recovery
state for $\code{f}_2$, since $\code{f}_1$ does not modify the field
$x$.  When the created recovery state reaches the load instruction at
line~6, it identifies it as a dependent load. Chopped symbolic
execution then creates another recovery state which executes $f_1$.
Once the recovery of $f_1$ is terminated, we can continue with the
recovery of $f_2$.

To make the symbolic execution more efficient in these cases, we
maintain for each state a cache which records the recovery states
which were actually relevant for any given load. By doing this, we can
avoid redundant recovery executions.

\begin{figure}[tbp]
\lstinputlisting[linewidth=.95\columnwidth]{code/multiple-skipped.c}
\caption{Multiple skipped functions.}\vspace{-5mm}
\label{fig:multiple-skipped-functions}
\end{figure}

%%% Local Variables:
%%% mode: latex
%%% TeX-master: "paper"
%%% End:


%!TEX root=./thesis.tex

\section{Memory allocations}
\label{Se:Malloc}

Let us consider the example from
Figure~\ref{fig:slices-with-allocations}, where the skipped function
\code{f} allocates memory with \code{malloc}. After skipping the
function call at line~\ref{line:alloc_skip}, the chopped symbolic
execution encounters two dependent loads at
lines~\ref{line:alloc_first} and \ref{line:alloc_second}. Chopped
symbolic execution thus spawns two consecutive recovery states: one in
which executes only line~\ref{line:slice_1}, and one in which executes lines~\ref{line:slice_1} and
lines~\ref{line:slice_2}. If we allowed \code{malloc} to return two different addresses while
executing the recovery states, this may lead to undefined behavior
since the second recovery would write to a different memory
address. To prevent this, and maintain consistency across recovery
states originating from the same function call, we maintain a list of
returned addresses for each allocation site in \code{f} which are
identified by their call stack. This way, subsequent recovery states
will use this information while re-executing allocating instructions.

\begin{figure}[tbp]
  \lstinputlisting[linewidth=.95\columnwidth]{code/allocations.c}\vspace{-2mm}
  \caption{Example of skipped function with allocation.}\vspace{-4mm}
\label{fig:slices-with-allocations}
\end{figure}

%%% Local Variables:
%%% mode: latex
%%% TeX-master: "paper"
%%% End:


%!TEX root=./paper.tex

\subsection{Chopping-Aware Search Heuristics}
\label{Se:Search}

Search heuristics are the main approach to reduce path explosion and
steer symbolic execution to uncovered paths for a more effective
exploration~\cite{exe,klee,sen:concolicheuristics,fitsymex:dsn09}, and
chopped symbolic execution is no exception. However, these heuristics
do not take into account the particular nature of the states in
chopped symbolic execution, that is the distinction between normal and
recovery states.

We propose a \textit{chopping-aware} search heuristic, which attempts
to optimize the exploration of chopped symbolic execution, while
aiming to a faster recovery of side effects. The heuristic's behavior
crucially depends on the current state being executed. Under normal
conditions---that is while symbolically executing normal states---the
search heuristic favors the selection of normal states at the next
steps. The rationale is that we want to favor paths that do not
require any recovery, thus fostering code exploration.

Instead, when chopped symbolic execution is executing a recovery
state, the search heuristic only selects the recovery states involved
in the recovery process. In this case, we want to favor a fast
recovery of the side effects needed by the dependent state. As soon as
the recovery state terminates, the search heuristic will switch to the
normal behavior.

Since always favoring normal states over recovery states may lead to
saturation in code exploration, we allow the searcher to select at a
lower probability a recovery state even under normal execution
conditions.

%%% Local Variables:
%%% mode: latex
%%% TeX-master: "paper"
%%% End:


%%% Local Variables:
%%% mode: latex
%%% TeX-master: "paper"
%%% End:
