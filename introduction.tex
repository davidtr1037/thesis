%!TEX root=./thesis.tex

\chapter{Introduction}\label{chapter:introduction}

Symbolic execution lies at the core of many modern techniques to
software testing, automatic program repair, and reverse
engineering~\cite{dart,cute,klee,semfix,chipounov2010reverse,JPF-SE}. At
a high-level, symbolic execution systematically explores multiple
paths in a program by running the code with symbolic values instead of
concrete ones. Symbolic execution engines thus replace concrete
program operations with ones that manipulate symbols, and add
appropriate constraints on the symbolic values. In particular,
whenever the symbolic executor reaches a branch condition that depends
on the symbolic inputs, it determines the feasibility of both sides of
the branch, and creates two new independent \emph{symbolic states}
which are added to a worklist to follow each feasible side
separately. This process, referred to as \emph{forking}, refines the
conditions on the symbolic values by adding appropriate constraints on
each path according to the conditions on the branch, in what is
referred to as the \textit{path condition}.  Test cases are generated
by finding concrete values for the symbolic inputs that satisfy the
path conditions.  To both determine the feasibility of path conditions
and generate concrete solutions to the path conditions, symbolic
execution engines employ \emph{satisfiability-modulo theory} (SMT)
constraint solvers~\cite{smt:cacm11}.

\section{The Challenge}
Symbolic execution has proven to be effective at finding subtle bugs
in a variety of software~\cite{klee,exe,JPF-SE,pex,sage}, and has
started to see industrial
take-up~\cite{symex-impact-11,sage,mayhem12}. However, a key remaining
challenge is scalability, particularly related to constraint solving
cost and path explosion~\cite{symex:cacm}.

Symbolic execution engines issue a huge number of queries to the
constraint solver that are often large and complex when analyzing
real-world programs. As a result, constraint solving dominates runtime
for the majority of non-trivial
programs~\cite{klee-multisolver,incremental-smt:hvc14}. Recent
research has tackled the challenge by proposing several constraint
solving optimizations that can help reduce constraint solving
cost~\cite{exe,cute,constraint-opt:cstva11,memoized:symex,
  green,greentrie,recal,klee-multisolver,klee-array}.

Path explosion represents the other big challenge facing symbolic execution,
and the main focus of this thesis.
Path explosion refers to the challenge of navigating the huge number of paths in real programs,
which is usually at least exponential in the number of static branches in the code.
The common mechanism employed by symbolic executors to
deal with this problem is the use of search heuristics to prioritize path exploration.
One particularly effective heuristic focuses on
achieving high statement and branch coverage by guiding the
exploration towards the path closest to uncovered
instructions~\cite{exe,klee,sen:concolicheuristics,fitsymex:dsn09}.
In practice, these heuristics only partially alleviate the path
explosion problem, as the following example demonstrates.

\section{Motivating Example.}
The \texttt{\_asn1\_extract\_der\_octet} function, shown in \Cref{fig:intro-1},
is taken from the \texttt{libtasn1} library (https://www.gnu.org/software/libtasn1)
which parses ASN.1 encoding rules from an input string.
The ASN.1 protocol is used in many networking and
cryptographic applications, such as those handling public key
certificates and electronic mail.
Versions of \texttt{libtasn1} before 4.5 are affected by a heap-overflow 
security vulnerability \footnote{CVE-2015-3622: https://cve.mitre.org/cgi-bin/cvename.cgi?name=CVE-2015-3622}
that could be exploited via a crafted certificate
Unfortunately, given a time budget of 24 hours,
the analysis of the \texttt{\_asn1\_extract\_der\_octet} function using
the state-of-the-art symbolic execution engine KLEE~\cite{klee} fails
to identify the vulnerability due to path explosion.

At each loop iteration (lines~\ref{line:loop_begin}--\ref{line:loop_end}),
the function decodes the current data element with \texttt{asn1\_get\_length} (line~\ref{line:heap_overflow}).
The function \texttt{asn1\_get\_length} parses the input string and decodes the length of the current component.
Then, the execution either recursively iterates over the input string (line~\ref{line:recursion}),
or invokes \texttt{\_asn1\_append\_value} (line~\ref{line:append}).
The function \texttt{\_asn1\_append\_value} (see~\Cref{fig:intro-2}) creates the
actual node in the Abstract Syntax Tree (AST) by decoding the input
string given the obtained length. This function scans once more over
the input string, performs several checks over the selected element,
and allocates memory for the node in the recursive data structure.

Path explosion in this function occurs due to several nested function
calls. Symbolically executing function \texttt{asn1\_get\_length} alone with
a symbolic string of \textit{n} characters leads to $4*n$ different paths.
Function \texttt{\_asn1\_append\_value} increases even more the number
of paths and also affects the efficiency of the symbolic execution
engine due to a huge number of constraint solver invocations. As a
result, the symbolic executor fails to identify the heap-overflow
vulnerability at line~\ref{line:heap_overflow}.

\begin{figure}
\lstinputlisting{code/intro-1.c}
\caption{
A simplified excerpt from the \texttt{\_asn1\_der\_extract\_octet} routine
in \texttt{libtasn1}. The invocation of \texttt{asn1\_get\_length\_der} in
line~\ref{line:get_length} leads to a heap overflow because
\texttt{der\_len} has not been decremented before the call.
}
\label{fig:intro-1}
\end{figure}

\begin{figure}
\lstinputlisting{code/intro-2.c}
\caption{A simplified excerpt from the \texttt{\_asn1\_append\_value}}
\label{fig:intro-2}
\end{figure}

\section{Our Approach}
Identifying the vulnerability from the entry point of the library is not trivial.
To reach the faulty invocation of function \texttt{asn1\_get\_length},
symbolic execution requires to analyze at the very least 2,945 calls
to 98 different functions, for a total amount of 386,727 executable
instructions. Our key observation is that most of the functions
required during the execution are \textit{not relevant} for finding
the vulnerability. The vulnerability occurs due to an incorrect update
of the remaining bytes for parsing (line~\ref{line:bug}), which
results in a memory out-of-bound read when calling
\texttt{asn1\_get\_length}. The bug thus occurs in code which deals with the
parsing, which is independent from the functions that constructs the
corresponding ASN.1 representation, such as \texttt{\_asn1\_append\_value}.
Therefore, we could have reached the bug even if we had \emph{skipped}
the irrelevant functions that build the AST.

In this thesis, we propose a novel form of symbolic execution called
\emph{chopped symbolic execution} that provides the ability to specify
parts of the code to exclude during the analysis, thus enabling
symbolic execution to focus on significant paths only. The skipped
code is not trivially excluded from symbolic execution, since this may
lead to spurious results. Instead, chopped symbolic execution lazily
executes the relevant parts of the excluded code when explicitly
required. In this way, chopped symbolic execution does not sacrifice
the soundness guarantees provided by standard symbolic
execution---except for non-termination of the skipped functions, which
may be considered a bug on its own---in that only feasible paths are
explored, but effectively discards paths irrelevant to the task at
hand.

We develop a prototype implementation of chopped symbolic
execution. We then report the results of an initial experimental
evaluation that demonstrates that this technique can indeed lead to
efficient and effective exploration of the code under analysis.

\section{Main Results}
In summary, in this thesis we make the following contributions:
\begin{enumerate}[leftmargin=*]
\item We introduce \emph{chopped symbolic execution}, a novel form of
  symbolic execution that leverages a lightweight specification of
  uninteresting code parts to significantly improve the scalability of
  symbolic execution, without sacrificing soundness.

\item We present \toolname, a prototype implementation of our
  technique within KLEE~\cite{klee}, and make it publicly available.

\item We report on an experimental evaluation of \toolname in two
  different scenarios: failure reproduction and test suite
  augmentation, and show that chopped symbolic execution can
  effectively improve upon standard symbolic execution in both cases.
\end{enumerate}


%%% Local Variables:
%%% mode: latex
%%% TeX-master: "thesis"
%%% End:
