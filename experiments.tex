%!TEX root=./paper.tex

\chapter{Experiments}\label{chapter:experiments}

Our evaluation aims to provide preliminary evidence that this novel
form of symbolic execution can lead to significant scalability
gains. In particular, we evaluate its effectiveness when embodied in
the following two scenarios:

\begin{enumerate}[leftmargin=*]
\item \textbf{Failure reproduction}, where the research question we
  explore is: How does chopped symbolic execution perform with respect
  to standard symbolic execution in generating an input that triggers
  a failure? In particular, can it reproduce more failures than standard
  symbolic execution, or can it reproduce the same failures faster?

\item \textbf{Test suite augmentation}, where the research question we
  explore is: How does chopped symbolic execution perform when steered
  to generate test cases that improve the structural coverage of code?
  Can chopped symbolic execution complement the exploration of
  standard symbolic execution?
\end{enumerate}

Note that our objective is not to claim that chopped symbolic
execution is generally a superior technique for a specific task---and
thus omits a direct comparison with other state-of-the-art techniques
for each scenario---but rather to assess the attainable benefits of
chopped symbolic execution when applied to techniques built upon
symbolic execution engines.

We compare \toolname with baseline KLEE. We use the same KLEE commit
(\texttt{b2f93ff}) from which we based \toolname.  Both tools are
compiled with LLVM 3.4.2~\cite{llvm} and use STP 2.1.2 as the
constraint solver~\cite{stp}. We conduct our experiments on two
identical servers running Ubuntu 14.04, equipped with an 8-core Intel
processor at 3.5 GHz and 16GB of RAM.

\section{Failure Reproduction}\label{sec:disc-secur-vuln}
In this experiment we use chopped symbolic execution for failure
reproduction. In particular, we run a symbolic executor to generate
inputs that trigger known security vulnerabilities.

\begin{table}[tbp]
  \caption{Security vulnerabilities in \textsf{libtasn1} considered
    for reproduction, with version used and number of lines of code.}
\begin{tabular}{|l|c|r|}
  \hline
  \textbf{Vulnerability} & \textbf{Version} & \textbf{C SLOC} \\ \hline
  CVE-2012-1569 & 2.11& 24,448 \\ \hline
  CVE-2014-3467 & 3.5 & 22,091 \\ \hline
  CVE-2015-2806 & 4.3 & 28,115 \\ \hline
  CVE-2015-3622 & 4.4 & 28,109 \\ \hline
\end{tabular}
\label{tab:vuln-stats}
\end{table}

\subsection{Benchmarks}
The subjects of this part of the evaluation are
vulnerabilities taken from GNU \textsf{libtasn1}. As briefly discussed
in the introduction, GNU \textsf{libtasn1} is a library for
serializing and deserializing data in Abstract Syntax Notation One
(ASN.1) format.  For example, \textsf{libtasn1} is used in GnuTLS to
define X.509 certificates.  We selected the \textsf{libtasn1} library
because its code is complex, with nested and deep function calls, and
can be successfully analyzed by the KLEE symbolic executor.
Table~\ref{tab:vuln-stats} lists and briefly characterizes the
vulnerabilities selected for our experiment, which are memory
out-of-bounds accesses.  Note that each vulnerability requires the
reproduction of a single failure, except for CVE-2014-3467, for which
the vulnerability can be exploited in three different code locations,
so we consider three different failures.  Therefore, in this
experiment we aim to reproduce a total of six failures.

\begin{table}[t]
  \caption{Results for the failure reproduction experiment for KLEE
    and \toolname on libtasn1. For each case we report the number of
    successful failures reproduced and the time required for the
    exploration (in \textit{hh:mm:ss}).}
\begin{tabular}{|l|C{1.4cm}|R{1.4cm}|R{1.4cm}|}
  \hline
  \textbf{Vulnerability} & \textbf{Search} & \makecell[c]{\textbf{KLEE}} & \makecell[c]{\textbf{\toolname}} \\ \hline\hline

  \multirow{3}{*}{CVE-2012-1569}
   & Random   & $\dagger$~11:52 & 02:27 \\ \cline{2-4}
   & DFS      & $\dagger$~05:08 & 03:29 \\ \cline{2-4}
   & Coverage & $\dagger$~11:28 & 02:45 \\
  \hline\hline

  \multirow{3}{*}{CVE-2014-3467$^1$}
   & Random   & 00:05 & 00:45 \\ \cline{2-4}
   & DFS      & 16:31 & 00:08 \\ \cline{2-4}
   & Coverage & 00:03 & 00:58 \\
  \hline\hline

  \multirow{3}{*}{CVE-2014-3467$^2$}
   & Random   &          1:02:13 & 06:18 \\ \cline{2-4}
   & DFS      & \textit{Timeout} & 00:09 \\ \cline{2-4}
   & Coverage &          1:33:56 & 02:48 \\
  \hline\hline

    \multirow{3}{*}{CVE-2014-3467$^3$}
   & Random   & \textit{Timeout} & 09:55 \\ \cline{2-4}
   & DFS      & \textit{Timeout} & 12:31 \\ \cline{2-4}
   & Coverage & \textit{Timeout} & 09:50 \\
  \hline\hline

  \multirow{3}{*}{CVE-2015-2806}
   & Random   &      1:07:46 & 02:18 \\ \cline{2-4}
   & DFS      &      2:46:13 & 12:04 \\ \cline{2-4}
   & Coverage &$\dagger$~38:56 & 01:02 \\
  \hline\hline

  \multirow{3}{*}{CVE-2015-3622}
   & Random   &  \textit{Timeout} & 00:16 \\ \cline{2-4}
   & DFS      &  \textit{Timeout} & 18:41 \\ \cline{2-4}
   & Coverage & 20:25:20 & 00:18 \\
  \hline
\end{tabular}
\flushleft
\begin{tabular}{L{\columnwidth}}
  \textit{Timeout}: Execution terminated after 24 hours \\
  $\dagger$ Execution terminated due to memory exhaustion \\
\end{tabular}\vspace{-5mm}
\label{tab:vulnerability-table}
\end{table}

\begin{table*}[t]
  \caption{Detailed results of \toolname for the failure reproduction
    experiment on libtasn1.}
{\def\arraystretch{0.9}
\begin{tabular}{|L{2.2cm}|C{1.5cm}|R{1.5cm}|R{1.6cm}|rr|rrr|}
  \hline
  \multirow{2}{*}{\textbf{Vulnerability}} & \multirow{2}{*}{\textbf{Search}} & \multirow{2}{*}{\makecell[c]{\textbf{Snapshots}}} & \multirow{2}{*}{\makecell[c]{\textbf{Recoveries}}} & \multicolumn{2}{c|}{\textbf{Function}} & \multicolumn{3}{c|}{\textbf{Slice}} \\
  & & & & \makecell[c]{\textbf{Full $\mathcal{F}$}} & \makecell[c]{\textbf{Sliced $\mathcal{F}$}} & \textbf{Num} & \makecell[c]{\textbf{$\mathcal{F}$ Size}} & \makecell[c]{\textbf{$\mathcal{S}$ Size}}\\ \hline\hline

  \multirow{3}{*}{CVE-2012-1569}
   & Random   & 5,315 & 7,447    & 01:21 & 02:27 & 4 & 694 & 320 (46\%)\\ \cline{2-9}
   & DFS      &   381 & 1,078    & 00:10 & 03:29 & 4 & 694 & 320 (46\%)\\ \cline{2-9}
   & Coverage & 6,258 & 9,053    & 01:53 & 02:45 & 4 & 694 & 320 (46\%)\\ \cline{2-9}
  \hline\hline

  \multirow{3}{*}{CVE-2014-3467$^1$}
   & Random   &  6,607 & 6,883   &00:30 &00:45 & 4 & 3,740 & 2,318 (62\%) \\ \cline{2-9}
   & DFS      &    656 & 1,003   &00:07 &00:08 & 4 & 3,740 & 2,318 (62\%) \\ \cline{2-9}
   & Coverage &  5,642 & 7,357   &00:50 &00:58 & 4 & 3,740 & 2,318 (62\%) \\ \cline{2-9}
  \hline\hline

  \multirow{3}{*}{CVE-2014-3467$^2$}
   & Random   & 16,279 & 26,300   &07:38 &06:18 & 4 & 3,740 & 2,318 (62\%) \\ \cline{2-9}
   & DFS      &    656 &  1,003   &00:11 &00:09 & 4 & 3,740 & 2,318 (62\%) \\ \cline{2-9}
   & Coverage & 10,147 & 18,916   &04:43 &02:48 & 4 & 3,740 & 2,318 (62\%) \\ \cline{2-9}
  \hline\hline

    \multirow{3}{*}{CVE-2014-3467$^3$}
   & Random   & 26,762 & 43,480   &17:30 &09:55 & 4 & 3,740 & 2,318 (62\%) \\ \cline{2-9}
   & DFS      & 38,696 & 61,113   &17:07 &12:31 & 4 & 3,740 & 2,318 (62\%) \\ \cline{2-9}
   & Coverage & 30,947 & 42,797   &17:13 &09:50 & 4 & 3,740 & 2,318 (62\%) \\ \cline{2-9}
  \hline\hline

  \multirow{3}{*}{CVE-2015-2806}
   & Random   &   173,065 & 0   & 02:31 & 02:18 & - & - & -\\ \cline{2-9}
   & DFS      & 2,708,849 & 0   & 12:30 & 12:04 & - & - & -\\ \cline{2-9}
   & Coverage &    36,549 & 0   & 01:04 & 01:02 & - & - & -\\ \cline{2-9}
  \hline\hline

  \multirow{3}{*}{CVE-2015-3622}
   & Random   &    584 &  8,980   & 00:25 & 00:16 & 6 & 1,269 & 343 (27\%)\\ \cline{2-9}
   & DFS      & 23,846 & 20,188   & 21:24 & 18:41 & 7 & 1,453 & 398 (27\%)\\ \cline{2-9}
   & Coverage &    608 &  9,043   & 00:23 & 00:18 & 6 & 1,269 & 343 (27\%)\\ \cline{2-9}
  \hline
\end{tabular}}
\label{tab:vulnerability-table-chaser}
\end{table*}

%%% Local Variables:
%%% mode: latex
%%% TeX-master: "../paper"
%%% End:


\subsection{Methodology}
We proceed with the following evaluation process:

\begin{enumerate}[leftmargin=*]
\item We manually create an execution driver for the \textsf{libtasn1}
  library to exercise the library from its public interface,
  simulating the interactions of an external program (\eg GnuTLS).
\item We manually derive the set of functions to skip by inspecting
  the code, and the vulnerability report that usually includes the
  stack trace and sometimes results from a dynamic analysis tool (\eg
  Valgrind~\cite{valgrind2003}). For the selected case studies we
  managed to identify a candidate set of function to exclude in less
  than 30 minutes per failure, but a developer familiar with the code
  should be able to do so faster.
\item We invoke KLEE and \toolname on the subject with several
  different search heuristics (random, DFS, and coverage-based) and a
  time limit of 24 hours. We also configure the symbolic executors to
  terminate the execution as soon as the vulnerability is
  identified. We do that by adding a new option to KLEE that, given a
  list of code locations, terminates the execution as soon as a
  vulnerability is discovered at all locations.
\end{enumerate}

Note that we initialize the pseudo-random number generator of the
random heuristic with the same seed, thus obtaining the same
exploration between consecutive runs and allowing for a fair
comparison between KLEE and \toolname.

\subsection{Results}
Table~\ref{tab:vulnerability-table} summarizes the high-level results
of our failure reproduction experiment. For each vulnerability and
search heuristic we report the effectiveness of KLEE and \toolname at
reproducing the failure as the time required to generate an input that
triggers the vulnerability.
% For \toolname we report for each
% vulnerability the best result among all validated configurations (more
% details later).

As can be seen, \toolname outperforms KLEE on all but one case study
both in terms of number of failures reproduced and performance,
regardless of the search heuristic applied. Overall, KLEE reproduces
four failures, CVE-2014-3467$^1$, CVE-2014-3467$^2$, CVE-2015-2806 and
CVE-2015-3622, and only failure CVE-2014-3467$^1$ can be reproduced
with all search heuristics. This latter case seems to be relatively
easy to identify, since KLEE requires only a few seconds. On the other
cases, KLEE requires between 1 and 20 hours. The problem of path
explosion in KLEE is particularly visible in CVE-2012-1569 where the
symbolic executor quickly runs out of available memory (4096 MB) and
thus fails to reproduce the failure.

In contrast, \toolname can identify all vulnerabilities and generates
a test case to reproduce each failure in less than 20 minutes, and
often much faster. Overall, for the vulnerabilities that KLEE can also
reproduce, CHASER can significantly beat KLEE in terms of performance
by at least an order of magnitude, with the only exception of
CVE-2014-3467$^1$ where \toolname can be slowed by the cost of static
analyses.

Table~\ref{tab:vulnerability-table-chaser} summarizes the detailed
results of \toolname for the failure reproduction experiment. For each
vulnerability and search heuristic we report the number of snapshots
and recovery states generated during chopped symbolic execution
(Snapshots and Recoveries, respectively), the execution times for
\toolname with and without slicing (Sliced $\mathcal{F}$ and Full
$\mathcal{F},$ respectively) as well as statistics on the generated
slices, which includes the number of slices generated (Num), and the
total size of the original ($\mathcal{F}$ size) and sliced
($\mathcal{S}$ size) skipped functions in terms of LLVM instructions.

Table~\ref{tab:vulnerability-table-chaser} shows that the number of
skipped functions (as deduced by the number of snapshot states) and
recovery states varies with the nature of the case study, the skipped
functions, and the search heuristic. In the case of vulnerability
CVE-2015-2806, \toolname could reproduce the failure without
recovering. This is the exemplar case that highlights the benefits of
chopped symbolic execution: While KLEE spent hours interpreting code
unrelated with the failure, \toolname excluded the uninteresting code
portions and could proceed analyzing only code of interest,
consistently identifying the failure with all search heuristics in as
little as one minute.

Table~\ref{tab:vulnerability-table-chaser} also shows that the benefit
of slicing the skipped functions depends on the case study. For
example, for the CVE-2014-3467$^3$ vulnerability, \toolname is on
average 70\% faster when slicing the skipped functions. Conversely,
\toolname performs the best without slicing in CVE-2012-1569. A
plausible explanation is that the additional analyses required for
slicing were more expensive than directly analyzing the functions. We
plan to develop a lightweight analysis to speculatively identify when
to apply slicing on the skipped functions.

\section{Test Suite Augmentation}
\label{sec:code-cover-impr}
In this experiment we use chopped symbolic execution for test suite
augmentation. We do that by running \toolname on a subject program
where we skip functions already exercised by an existing test
suite. As initial test suite we rely on tests generated by KLEE. In
essence, we want to assess the effectiveness of chopped symbolic
execution in complementing standard symbolic execution in test
generation, ultimately increasing structural coverage.

\subsection{Benchmarks}
The subjects of this part of the evaluation are
GNU \textsf{BC} 2.27 and \textsf{LibYAML}. \textsf{BC} is an
arbitrary-precision calculator that solves mathematical expressions
written in a C-style language.  \textsf{LibYAML} is a well-known
library for parsing and emitting data in YAML format, which is a human
friendly data serialization standard.  We choose \textsf{BC} and
\textsf{LibYAML} because KLEE has a hard time generating high-coverage
tests. As a result, the code not covered by KLEE is usually related to
complex features, and we challenge \toolname to exercise it. For each
program, we rely on the program's documentation and personal
experience with the subject to identify the best argument
configuration that can maximize coverage.

\subsection{Methodology}
We proceed with the following evaluation process:

\begin{enumerate}[leftmargin=*]
\item We generate the initial test suite by running KLEE on each
  subject with coverage-based search heuristic and a time limit of one
  hour. We use this configuration to maximize structural coverage of
  the code under analysis, in particular we focus on statement and
  branch coverage.
\item We compute the structural coverage obtained with the test suites
  that KLEE generates using GNU
  GCov.\footnote{https://gcc.gnu.org/onlinedocs/gcc-4.7.3/gcc/Gcov.html}
\item We use the coverage information and the call graph to select for
  each program the set of functions to skip based on the coverage
  information. For example, suppose that function $f$ calls function
  $g$ and $h$, and that $f$ and $h$ are covered by a test. We include
  in the set of the skipped functions only $h$, since $f$ is required
  to reach uncovered function $g$.
\item We invoke \toolname on the subjects with coverage-based search
  heuristic and a time limit of one hour.
\end{enumerate}

%!TEX root=../thesis.tex

\begin{table}
\caption{
Line (\textbf{L}) and branch (\textbf{B}) coverage achieved by KLEE
and \toolname on \textsf{BC} and \textsf{libyaml} in one hour.
}
\begin{tabular}{|l|rr|rr|rr|}
  \hline
  \multirow{3}{*}{\textbf{Program}} & \multicolumn{2}{c|}{\multirow{2}{*}{\textbf{KLEE}}} & \multicolumn{4}{c|}{\textbf{KLEE+\toolname}} \\
  & & & \multicolumn{2}{c|}{\textbf{Full $\mathcal{F}$}} & \multicolumn{2}{c|}{\textbf{Sliced $\mathcal{F}$}} \\
  & \makecell[c]{\textbf{L}} & \makecell[c]{\textbf{B}} & \makecell[c]{\textbf{L}} & \makecell[c]{\textbf{B}} & \makecell[c]{\textbf{L}} & \makecell[c]{\textbf{B}} \\\hline
  BC      & 12.52\% & 10.16\% & 14.73\% & 12.68\% & 14.95\% & 13.31\% \\ \hline
  LibYAML & 10.8\%  & 4.2\%   & 22.6\%  & 15.0\%  & 22.6\%  & 15.0\% \\
  \hline
\end{tabular}
\label{tab:experiments-table-augmentation}
\end{table}

%%% Local Variables:
%%% mode: latex
%%% TeX-master: "../thesis"
%%% End:


\subsection{Results}
Table~\ref{tab:experiments-table-augmentation} summarizes the results
of our test suite augmentation experiment. For each case study we
report the structural coverage of a symbolic executor as percentage of
statements (Lines) or branches (Branches) covered by its generated
test suite. For CHASER we report the structural coverage results with
and without performing slicing (Sliced $\mathcal{F}$ and Full
$\mathcal{F}$, respectively).

Table~\ref{tab:experiments-table-augmentation} shows that \toolname
effectively complements KLEE and increases code coverage even on
complex subjects. Specifically, on \textsf{BC}, \toolname increased
statement and branch coverage by +2.43\% and +3.16\%, respectively,
and on \textsf{LibYAML} it more than doubled statement coverage (from
10.8\% to 22.6\%) and more than tripled branch coverage (from 4.2\% to
15\%).

In \textsf{LibYAML} we observed that KLEE spent almost all its budget
analyzing one function that contains complex logic responsible for
ensuring that the buffer contains enough characters for parsing while
handling different encodings, such as UTF-8 or UTF-16. Such function
is invoked at the beginning of program execution, and KLEE got stuck
into it, not being able to execute any subsequent line of
code. Conversely, \toolname skipped the expensive invocation and
continued to explore other parts of the code. Our chopping-aware
search heuristic also allowed us to recover paths inside the expensive
function while giving higher priority to non-recovery states, in turn
resulting in a more in-depth exploration of the code.

In case of \textsf{BC}, \toolname managed to skipped expensive
functions that initialize the parsing of the input file and reached
the actual parsing functions. Unfortunately, the analysis quickly got
stuck in the parsing routine due to timeouts in the constraint solver
due to the complex constraints, resulting in a limited increase in
coverage.

As for the previous experiment, the benefit of slicing strictly depend
on the case study. In this experiment, slicing is not beneficial in
\textsf{LibYAML}, while it contributes to increase coverage on
\textsf{BC}.

\section{Threats to Validity}
Here we briefly discuss the countermeasures we adopted to mitigate the
threats to validity. The internal validity depends on the correctness
of our prototype implementations, and may be threatened by the
evaluation setting and the execution of the experiments.  We carefully
tested our prototype with respect to the original KLEE baseline, and
make it available for further inspection.

Threats to external validity may derive from the selection of
benchmarks. We validated our approach on three real-world
subjects. Different results could be obtained for different
subjects. The only way to further reduce the external validity threat
consists in replicating our study on more subjects. For this reason we
make our experimental package publicly available to other
researchers.\footnote{http://url-omitted-for-double-blind-revision}

%%% Local Variables:
%%% mode: latex
%%% TeX-master: "paper"
%%% End:
