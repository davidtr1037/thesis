%!TEX root=./paper.tex

\chapter{Related Work}\label{chapter:related}

The research community has invested significant effort in addressing
the path explosion challenge in symbolic execution, and this paper
aligns with this line of work.

As we already mentioned in the introduction, the most common and often
most effective mechanism employed by symbolic executors are search
heuristics, whose goal is to guide program exploration to the most
promising paths in the program.  Popular heuristics include random
path exploration~\cite{klee}, generational search~\cite{sage} and
coverage-optimized search~\cite{exe,sen:concolicheuristics}, to name
just a few.  Unfortunately, search heuristics only partly alleviate
path explosion, and symbolic execution can still get stuck in
irrelevant parts of the code.

Another effective technique is to try to prune equivalent program
paths~\cite{exe:tacas,rwset2}.  For instance, if a path reaches a
program point with a set of constraints equivalent to those of a
previous path that reached that point, then the second path (and all
paths that it would have spawned) can be terminated.  This technique
is similar in spirit to our approach, but orthogonal, as it does
nothing to prevent the exploration of code irrelevant to the task at
hand.  Chopped symbolic execution can be combined with path pruning,
in order to prune both irrelevant paths, as well as those relevant
paths which are equivalent to other relevant paths.

Merging paths can also help alleviate path explosion.  Paths can be
merged either ahead-of-time~\cite{klee-fp,kleecl:tse} or at
runtime~\cite{merging:pldi12,multise:fse15}. A particular type of path
merging are function summaries, in which paths within a function are
merged into a summary that can be reused on subsequent
invocations~\cite{godefroid:popl,godefroid:tacas}. Path merging can
lead to exponential reduction in the number of paths explored, but the
cost is often off-loaded to the constraint solver, which has to deal
with significantly harder constraints.  Again, chopped symbolic
execution could be combined with path merging, in order to get the
benefit of both.

Chopped symbolic execution makes use of program slicing in order to
explore only the relevant parts of code through the skipped functions.
Program slicing has been explored in symbolic execution before, \eg in
the context of patch testing~\cite{babic11}.


%%% Local Variables:
%%% mode: latex
%%% TeX-master: "paper"
%%% End:
