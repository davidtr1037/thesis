%!TEX root=./thesis.tex

\chapter{Related Work}\label{chapter:related}

The research community has invested significant effort in addressing
the path explosion challenge in symbolic execution, and this thesis
aligns with this line of work.

As we already mentioned in the introduction, the most common and often
most effective mechanism employed by symbolic executors are search
heuristics, whose goal is to guide program exploration to the most
promising paths in the program.  Popular heuristics include random
path exploration~\cite{klee}, generational search~\cite{sage} and
coverage-optimized search~\cite{exe,sen:concolicheuristics}, to name
just a few.  Unfortunately, search heuristics only partly alleviate
path explosion, and symbolic execution can still get stuck in
irrelevant parts of the code.

Another effective technique is to try to prune equivalent program
paths~\cite{exe:tacas,rwset2}.  For instance, if a path reaches a
program point with a set of constraints equivalent to those of a
previous path that reached that point, then the second path (and all
paths that it would have spawned) can be terminated.  This technique
is similar in spirit to our approach, but orthogonal, as it does
nothing to prevent the exploration of code irrelevant to the task at
hand.  Chopped symbolic execution can be combined with path pruning,
in order to prune both irrelevant paths, as well as those relevant
paths which are equivalent to other relevant paths.

Merging paths can also help alleviate path explosion.  Paths can be
merged either ahead-of-time~\cite{klee-fp,kleecl:tse} or at
runtime~\cite{merging:pldi12,multise:fse15}. A particular type of path
merging are function summaries, in which paths within a function are
merged into a summary that can be reused on subsequent
invocations~\cite{godefroid:popl,godefroid:tacas}. Path merging can
lead to exponential reduction in the number of paths explored, but the
cost is often off-loaded to the constraint solver, which has to deal
with significantly harder constraints.  Again, chopped symbolic
execution could be combined with path merging, in order to get the
benefit of both.

Chopped symbolic execution makes use of program slicing in order to
explore only the relevant parts of code through the skipped functions.
Program slicing has been explored in the context of symbolic execution before ~\cite{babic11,slaby2013symbiotic}.
The \textit{Symbiotic}~\cite{slaby2013symbiotic} tool uses static slicing for
detecting bugs described by state machines,
by first slicing the program and then executing it symbolically.
Although the usage of slicing here is not the same as in our tool,
the approach of slicing the program ahead without any a priori information is likely to be less efficient.
If the side effects of a function are used only in some subset of the paths,
then slicing the program ahead will not be effecient,
as the static slicer consideres all the execution paths of the program.
In our tool, the lazy resolving of dependencies enables us to perform the slicing on demand,
only when the side effects are needed in a particular path in the program.
This way, the function is entirely ignored on paths which does not depend on it's side effects
and sliced on the other ones.

The \textit{Latset}~\cite{majumdar2007latest} tool uses lazy expansion for test input generation.
It first explores an abstration of the tested function,
by replacing each function call with an unconstrained input.
Then it tries to expand the abstrat path to a feasible path
by exapnding the called functions and finding concrete executions in them
which are consistent with the abstract path.
Introducing symbolic vales as a replacement for a function call
is not efficient when the values replace pointers,
as symbolic pointers are hard to handle in symbolic execution.
While dealing with function side effects,
our tool does not introduce additional symbolic values,
but rather recovers the missing values by concretley executing the relevant skipped functions.
Another major difference between the approaches is the way in which
the skipped function calls are expanded.
In \textit{Latest}, the whole path is re-executed in order to make sure that the
localy expanded paths are consistent with the abstract path.
Our tool executes only the skipped functions
and uses the concepct of \textit{guiding constraints} to ensure the consistency.


%%% Local Variables:
%%% mode: latex
%%% TeX-master: "thesis"
%%% End:
