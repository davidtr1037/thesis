%!TEX root=./thesis.tex

\section{Multiple Recovery States}
\label{Se:MultiRecovery}
In some cases, we need to create several recovery states during a
chopped symbolic execution with a single skipped function.
For example, consider the following code fragment:

\begin{figure}
  \captionsetup[subfloat]{labelformat=empty}
  \subfloat[]{
    \lstinputlisting[linewidth=.4\textwidth,numbers=none]{code/multiple-recovery-states.c}
  }
  \caption{Allocation sites in pointer analysis}
  \label{fig:multiple-recovery-states}
\end{figure}

In this example, we have two dependent loads:
one which reads from $\code{p->x}$ and another which reads from $\code{p->y}$.
As a result, two recovery states are expected to be created.
Assume that the first recovery state goes through the branch which assumes $k > 0$.
If the symbolic execution of the second recovery state goes through the path in which
$\code{p.y}$ is updated ($k\le0$), the induced combined execution
would be infeasible. To avoid this undesirable situation, when a
recovery state terminates, it adds the new constraints accumulated in
its path condition to the \textit{guiding constraints} of its dependent state. The added
constraints are then used in subsequent recovery states. In our
example in Figure~\ref{fig:multiple-recovery-states}, the constraint $k > 0$ is
propagated from the first recovery state to the first dependent state,
thus ensuring that the symbolic execution of the second recovery state
does not follow an infeasible path.

%%% Local Variables:
%%% mode: latex
%%% TeX-master: "thesis"
%%% End:
