%!TEX root=./paper.tex

\section{Multiple Recovery States}
\label{Se:MultiRecovery}
In some cases, we need to create several recovery states during a
single chopped symbolic execution.

For example, consider the following code fragment which modifies the
body of the \code{main()} function in Figure~\ref{fig:simple}:
\[
\code{f(\&p,k); if (p.x) \{ p.z++; \} if (p.y) \{ p.z--; \}}
\]
If we wish to skip the invocation to $\code{f}()$ then a recovery
state and a dependent state are created every time the condition of
the branch needs to be evaluated. Note that the second dependent state
is produced from the first dependent one and that the resumed state
encapsulates the changes made by the first recovery state. Assume that
these changes involve a modification of the value of $\code{p.x}$
inside the $k>0$ branch at line~\ref{line:px_write}. If the symbolic
execution of the second recovery state goes through the path in which
$\code{p.y}$ is updated ($k\le0$), the induced combined execution
would be infeasible. To avoid this undesirable situation, when a
recovery state terminates, it adds the new constraints accumulated in
its path condition to the \textit{guiding constraints} of its dependent state. The added
constraints are then used in subsequent recovery states. In our
example in Figure~\ref{fig:simple}, the constraint $k > 0$ is
propagated from the first recovery state to the first dependent state,
thus ensuring that the symbolic execution of the second recovery state
does not follow an infeasible path.

%%% Local Variables:
%%% mode: latex
%%% TeX-master: "paper"
%%% End:
