%!TEX root=./thesis.tex

\section{Handling Multiple Skipped Functions}
\label{Se:SkipMultipleFuncs}
So far, we have assumed that every symbolic state has a single skipped
invocation. When we have multiple invocations that can be skipped, we
then need to decide which functions to use for recovery and in which
order, in the case multiple invocations modify the dependent load
address $\addr$. We solve this issue by executing the skipped
invocations according to their order along the path, thus ensuring
that the value stored in $\addr$ at the end of the recovery process is
indeed the last value written there along the chopped path.

Another issue that we need to address to support multiple skipped
functions is that a skipped invocation might depend on the side
effects of an earlier skipped functions. When this happens, we apply
our recovery approach in a recursive manner, and treat the current
recovery state as a \emph{dependent} state. For example, consider the
code in Figure~\ref{fig:multiple-skipped-functions}. When the
execution reaches the dependent load at line~13, we create a recovery
state for $\code{f}_2$, since $\code{f}_1$ does not modify the field
$x$.  When the created recovery state reaches the load instruction at
line~6, it identifies it as a dependent load. Chopped symbolic
execution then creates another recovery state which executes $f_1$.
Once the recovery of $f_1$ is terminated, we can continue with the
recovery of $f_2$.

To make the symbolic execution more efficient in these cases, we
maintain for each state a cache which records the recovery states
which were actually relevant for any given load. By doing this, we can
avoid redundant recovery executions.

\begin{figure}[tbp]
\lstinputlisting[linewidth=.95\columnwidth]{code/multiple-skipped.c}
\caption{Multiple skipped functions.}\vspace{-5mm}
\label{fig:multiple-skipped-functions}
\end{figure}

%%% Local Variables:
%%% mode: latex
%%% TeX-master: "thesis"
%%% End:
