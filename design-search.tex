%!TEX root=./thesis.tex

\section{Chopping-Aware Search Heuristics}
\label{Se:Search}

Search heuristics are the main approach to reduce path explosion and
steer symbolic execution to uncovered paths for a more effective
exploration~\cite{exe,klee,sen:concolicheuristics,fitsymex:dsn09}, and
chopped symbolic execution is no exception. However, these heuristics
do not take into account the particular nature of the states in
chopped symbolic execution, that is the distinction between normal and
recovery states.

We propose a \textit{chopping-aware} search heuristic, which attempts
to optimize the exploration of chopped symbolic execution, while
aiming to a faster recovery of side effects. The heuristic's behavior
crucially depends on the current state being executed. Under normal
conditions---that is while symbolically executing normal states---the
search heuristic favors the selection of normal states at the next
steps. The rationale is that we want to favor paths that do not
require any recovery, thus fostering code exploration.

Instead, when chopped symbolic execution is executing a recovery
state, the search heuristic only selects the recovery states involved
in the recovery process. In this case, we want to favor a fast
recovery of the side effects needed by the dependent state. As soon as
the recovery state terminates, the search heuristic will switch to the
normal behavior.

Since always favoring normal states over recovery states may lead to
saturation in code exploration, we allow the searcher to select at a
lower probability a recovery state even under normal execution
conditions.

%%% Local Variables:
%%% mode: latex
%%% TeX-master: "search"
%%% End:
